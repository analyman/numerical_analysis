\chapter{插值法}

\section{拉格朗日插值}

利用$x_0,\cdots,x_n$ $n+1$个点有拉格朗日$n$阶插值多项式$L_n(x)$

\begin{align}
    L_n(x) = \sum_{i=0}^n\,f(x_i)l_i(x)
\end{align}

$l_i(x)$为插值基函数
\begin{align}
    l_i(x) = {(x-x_0)\cdots(x-x_{i-1})\cdot(x-x_{i+1})\cdots(x-x_n) \over (x_i-x_0)\cdots(x_i-x_{i-1})\cdot(x_i-x_{i+1})\cdots(x_i-x_n)}
\end{align}

拉格朗日插值多项式的余项$R_n(x) = f(x) - L_n(x)$, 若$f(x) \in C^{n+1}$, 则有
\begin{align}
    \label{equ:remainv}
    R_n(x) = {f^{(n+1)}(\xi)\over (n+1)!}\omega_{n+1}(x)
\end{align}

\fbox{\vtop{
    设$\displaystyle \phi(t) = f(t) - L_n(t) - {R_n(x)\over\omega_{n+1}(x)}\omega_{n+1}(t)$, $\phi(t)$有$x_0,\cdots,x_n,x$一共$n+2$个零点,
    所以$\exists\xi$, 使得
    $$\displaystyle\phi^{(n+1)}(\xi) = f^{(n+1)}(\xi) - L_n^{n+1}(\xi) - {R_n(x)\over \omega_{n+1}(x)}(n+1)! = 0$$
    即有
    $$R_n(x) = {f^{(n+1)}(\xi)\over (n+1)!}\omega_{n+1}(x)$$
}}\par
\smallskip
其中$\omega_{n+1}(x)$为
\begin{align}
\omega_{n+1}(x) = \prod_{i=0}^{n}(x-x_i)
\end{align}

特别有\(\displaystyle\sum_{i=0}^nx_i^k l_i(x) = x^k\), 
取$k=0$,既有\(\displaystyle \sum_{i=0}^n l_i(x) = 1\)

\section{牛顿插值}

\subsection*{均差}

\(\displaystyle f[x_i] = f(x_i)\)\par
\(\displaystyle f[x_i,x_j] = f[x_j,x_i] = {f(x_i) - f(x_j)\over x_i - x_j}\)\hfill 一阶均差\par
\(\displaystyle f[x_i,x_j,x_k] = f[x_j,x_i,x_k] = f[x_k,x_i,x_j] = \cdots = {f[x_i,x_j] - f[x_j,x_k]\over x_i - x_k} = \cdots\)\hfill 二阶均差\par
$\displaystyle\vdots$

牛顿插值多项式
\begin{align}
    P_n(x) &= f[x_0] + f[x_0,x_1](x-x_0) + \cdots + f[x_0,\cdots,x_n]\omega_n(x)\cr
    \omega_n(x) &= \prod_{i=0}^{n-1}(x-x_i)
\end{align}

余项和拉格朗日插值多项式的余项相同,
\begin{align}
    R_n(x) &= f(x) - P_n(x) = f[x_0,\cdots,x_n,x]\omega_{n+1}(x)\cr
           &= {f^{(n+1)}(\xi)\over(n+1)!}\omega_{n+1}(x)
\end{align}

可以看出
\begin{align}
    f[x_0,\cdots,x_n] = {f^{(n)}(\xi)\over n!}
\end{align}

\section{Hermite插值}

给形如$f^{(k)}(x_i) = y_i$的$m+1$个条件可以构造$m$次 Hermite 插值多项式。 三点三次 Hermite插值多项式的例子\par
\medskip
\fbox{\vbox{
    条件为$H(x_i) = f(x_i)\quad(i=0,1,2)$及$H'(x_1) = f'(x_1)$,求$H(x)$\par
    由$H(x_i) = f(x_i)\quad(i=0,1,2)$, 可设
    \begin{align*}
        H(x) = f[x_0] + f[x_0,x_1](x-x_0) + f[x_0,x_1,x_2](x-x_0)(x-x_1) + A(x-x_0)(x-x_1)(x-x_2)
    \end{align*}
    利用$H'(x_1) = f'(x_1)$可以得到
    \begin{align*}
        A = {f'(x_1) - f[x_0,x_1] - (x_1-x_0)f[x_0,x_1,x_2]\over (x_1-x_0)(x_1-x_2)}
    \end{align*}

    余项可以利用类似\ref{equ:remainv}的方法求出,
    \begin{align*}
        R_4(x) = f(x) - H(x) = {f^{(4)}(\xi)\over 4!}(x-x_0)(x-x_1)^2(x-x_2)
    \end{align*}
}}

\section{分段低次插值}

并不是多项式次数越高精度一定越好

\section{三次样条插值}

有$(x_i, f(x_i))\quad(i=0,\dots,n)$ $n+1$个点, 函数$S_n(x)$为满足
通过每一个点并且二阶可导的分段函数, 在每个$(x_i,x_{i+1})$区间$S_n(x)$都是三次多项式函数,
所以$S_n(x)$一共有$4n$个参数, $S_n(x)$的约束有\par
\hskip.3\hsize
\begin{tabular}{|c|c|}
    \hline
    约束 & 次数\\\hline
    $S(x_i) = f(x_i)$ & $n+1$\\\hline
    $S(x_i - 0) = S(x_i + 0)$ & $n-1$\\\hline
    $S'(x_i - 0) = S'(x_i + 0)$ & $n-1$\\\hline
    $S''(x_i - 0) = S''(x_i + 0)$ & $n-1$\\\hline
     & $4n-2$\\\hline
\end{tabular}

\medskip
所以需要额外的$2$个约束来确定$S_n(x)$的所有参数, 可以取$S''(x_0) = S''(x_n) = 0$称为自然边界条件
