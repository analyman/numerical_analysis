\chapter{线性方程组}

\section{三角分解法}

求$Ax=b$等价于求$Ly = b$与$Ux = y$, 其中$$A = LU$$
$L$为单位下三角矩阵, $U$为上三角矩阵

\begin{align}
\begin{pmatrix}
    a_{11} & a_{12} & \cdots & a_{1n}\cr
    a_{21} & u_{22} & \cdots & a_{2n}\cr
    \vdots & \vdots &        & \vdots \cr
    a_{n1} & a_{n2} & \cdots & a_{nn}\cr
\end{pmatrix}
 = \begin{pmatrix}
    1\cr
    l_{21} & 1\cr
    \vdots & \vdots & \ddots\cr
    l_{n1} & l_{n2} & \dots & 1\cr
\end{pmatrix}\cdot \begin{pmatrix}
    u_{11} & u_{12} & \cdots & u_{1n}\cr
           & u_{22} & \cdots & u_{2n}\cr
           &        & \ddots & \vdots \cr
           & & &               u_{nn}\cr
\end{pmatrix}
\end{align}

$l_{ij}$和$u_{ij}$的计算公式

\begin{align}
u_{ri} &= a_{ri} - \sum_{k=1}^{r-1}l_{rk}u_{ki}, \quad i= r,r+1,\cdots,n;\cr
l_{ir} &= {\displaystyle\left( a_{ir} - \sum_{k=1}^{r-1}l_{ik}u_{kr}\right)\over u_{rr}}, \quad i= r+1,\cdots,n, \text{且}r\neq n.
\end{align}

\begin{algorithm}
\begin{algorithmic}[1]
\For{$i=1,\ldots,n$}
    \State {求$U$的第$i$行}
    \State {求$L$的第$i$列}
\EndFor
\end{algorithmic}
\caption{矩阵三角分解(LU分解)}
\end{algorithm}

\section{矩阵的范数}

\halign{
    #\hfill\quad & #\hfill & #\hfill\cr
    $F$范数 & $\displaystyle F(A) = \norm{A}_F = \left(\sum_{i,j=1}^n\,a_{i,j}^2\right)^{1\over 2}$\cr
    行范数 & $\displaystyle\norm{A}_{\infty} = \max_{1\le i\le n}\sum_{j=1}^n|a_{ij}|$\cr
    列范数 & $\displaystyle\norm{A}_{1}      = \max_{1\le j\le n}\sum_{i=1}^n|a_{ij}|$\cr
    $2-$范数 & $\displaystyle \norm{A}_2 = \sqrt{\lambda_{\max}\left(A^TA\right)}$\cr
    算子范数 & $\displaystyle \norm{A}_v = \max_{x\neq 0}{\norm{Ax}_v\over \norm{x}_v}$ & $\norm{Ax}_v\le\norm{A}_v\norm{x}_v$,\quad$v$分别取$\infty,1,2$就是上面三种范数\cr
}

\section{迭代法}

\begin{align}
    x^{(k+1)} = Bx^{(k)} + f
    \label{equ:iterative_method}
\end{align}

\ref{equ:iterative_method}收敛的充要条件为\par
\smallskip
\hskip1.5em\hbox{\vbox{
\begin{easylist}
    # $\displaystyle\lim_{k\to\infty} B^k = 0$
    # $\displaystyle\rho(B) < 1, \quad\rho(B) = \max(|\lambda_1|,\cdots,|\lambda_n|)$
    # 存在$B$的算子范数使得$\norm{B}_v < 1$
\end{easylist}}}

\medskip
迭代法收敛速度$R(B) = -\ln\rho(B)$

\begin{align}A = D - L - U\end{align}
\begin{table}[H]
\hbox to\linewidth{\hfill\vbox{
		\halign{
			\vrule\vbox to1.3em{}\vtop to0.8em{}\quad#\hfill\quad &\vrule\quad#\hfill\quad\vrule &\quad #\hfill\quad\vrule\cr
			\noalign{\hrule}
            雅可比迭代法 & \hbox to.3\hsize{\vtop{\noindent\(B = D^{-1}(L + U)\) \newline \(f= D^{-1}b\)\smallskip}} & \cr
			\noalign{\hrule}
            高斯-赛德尔迭代法 & \hbox to.3\hsize{\vtop{\noindent$B = (D - L)^{-1}U$\newline $f = (D-L)^{-1}b$\smallskip}} &\cr
			\noalign{\hrule}
            超松弛迭代法(SOR) & \hbox to.4\hsize{\vtop{\noindent\(B\equiv L_\omega = (D-\omega L)^{-1}\left[(1-\omega)D + \omega U\right]\)\newline \(f = \omega(D-\omega L)^{-1}b\)\smallskip}} & $\omega$的取值范围为$(0,2)$({\bf 收敛的必要条件})\cr
			\noalign{\hrule}
		}
    }\hfill
}
\caption{对解线性方程组$Ax = b$,不同迭代法$B$的取值}
\end{table}

